%%
%% forked from https://gits-15.sys.kth.se/giampi/kthlatex kthlatex-0.2rc4 on 2020-02-13 expanded upon by Gerald Q. Maguire Jr. This template has been adapted by Anders
%% Sjögren to the University Engineering Program in Computer Science at KTH ICT. This adaptation was translation of English headings into Swedish with the addition of
%% Swedish. Many thanks to others who have provided constructive input regarding the template.

% Make it possible to conditionally depend on the TeX engine used
\RequirePackage{ifxetex}
\RequirePackage{ifluatex}
\newif\ifxeorlua
\ifxetex\xeorluatrue\fi
\ifluatex\xeorluatrue\fi

\ifxeorlua
% The following is to ensure that the PDF uses a recent version rather than the typical PDF 1-5. This same version of PDF should be set as an option for hyperef

\RequirePackage{expl3}
\ExplSyntaxOn
%pdf_version_gset:n{2.0}
%\pdf_version_gset:n{1.5}

%% Alternatively, if you have a LaTeX newer than June 2022, you can use the following. However, then you have to remove the pdfversion from hyperef. It also breaks
%% hyperxmp. So perhaps it is too early to try using it!
%\DocumentMetadata
%{
%% testphase = phase-I, % tagging without paragraph tagging
% testphase = phase-II % tagging with paragraph tagging and other new stuff.
%pdfversion = 2.0 % pdfversion must be set here.
%}

% Optionally, you can set the uncompress flag to make it easier to examine the PDF
%\pdf_uncompress: % to check the pdf
\ExplSyntaxOff
\else
\RequirePackage{expl3}
\ExplSyntaxOn
%\pdf_version_gset:n{2.0}
\pdf_version_gset:n{1.5}
\ExplSyntaxOff
\fi

%% Define a pair of commands to disable and reenable specific packages - see https://tex.stackexchange.com/questions/39415/unload-a-latex-package
\makeatletter
\newcommand{\disablepackage}[2]{%
\disable@package@load{#1}{#2}%
}
\newcommand{\reenablepackage}[1]{%
\reenable@package@load{#1}%
}
\makeatother
%% To avoid the warning: "Package transparent Warning: Loading aborted, because pdfTeX is not running in PDF mode."
\ifxeorlua
\disablepackage{transparent}{}
\fi

%% The template is designed to handle a thesis in English or Swedish set the default language to english or swedish by passing an option to the documentclass - this
%% handles the inside title page To optimize for digital output (this changes the color palette add the option: digitaloutput To use \ifnomenclature add the option
%% nomenclature To use bibtex or biblatex - include one of these as an option
\documentclass[nomenclature, english, bibtex]{kththesis}
%\documentclass[swedish, biblatex]{kththesis}
% if pdflatex \usepackage[utf8]{inputenc}
\usepackage{amsfonts}
\usepackage{amsbsy}
\usepackage{multirow}

\definecolor{mycustomgray}{gray}{0.65}
\newenvironment{incomplete}{%
\color{mycustomgray}%
}{%
}
%% Conventions for todo notes:
% Informational

\cleardoublepage
\chapter{Background}
\label{ch:background}

\section{Epilepsy}

A seizure represents a fundamental disruption of normal brain function, characterized by a transient surge of abnormal, excessive, or synchronous neuronal activity within
the brain~\cite{fisher2014iladef, fisher2017operational}. These paroxysmal events can manifest through a wide spectrum of signs and symptoms, the nature of which is
intrinsically linked to the specific cerebral networks involved in the ictal discharge and its subsequent propagation~\cite{thijs2019epilepsy}. While an isolated seizure
may be provoked by acute systemic insults or transient neurological stressors, the term epilepsy designates a more profound neurological condition: an enduring
predisposition to generate unprovoked epileptic seizures, along with the associated neurobiological, cognitive, psychological, and social ramifications~\cite{berg2010revised}.
Clinically, epilepsy is typically diagnosed after at least one unprovoked seizure, especially when factors suggest a high likelihood of recurrence.

The global impact of seizures and epilepsy is considerable. Affecting an estimated 50 million individuals worldwide, epilepsy stands as one of the most prevalent serious
neurological disorders~\cite{who2023epilepsy}. Beyond the immediate risk of physical injury during an ictal event, seizures can precipitate or exacerbate cognitive
difficulties, particularly in domains of memory and attention, and are frequently associated with psychological comorbidities such as anxiety and
depression~\cite{jobst2010cognitive, fiest2017depression}. The societal impact is also profound, with individuals often confronting stigma, discrimination, and limitations
in education, employment, and personal autonomy~\cite{deboer2008stigma}. The consequences of recurrent seizures permeate virtually every aspect of an individual's life.

% Clinically, the manifestations of seizures are strikingly diverse. The International League Against Epilepsy (ILAE) provides a foundational framework, classifying
% seizures based on their onset—focal, generalized, or unknown~\cite{fisher2017operational}. Focal onset seizures originate within networks confined to one cerebral
% hemisphere and can occur with or without impaired awareness, presenting with a range of motor or non-motor symptoms. In contrast, generalized onset seizures are presumed
% to involve widespread, bilateral networks from their inception, typically leading to immediate impairment of consciousness. The category of unknown onset is reserved for
% seizures where the initial point of origin cannot be confidently determined~\cite{fisher2017operational}. This broad spectrum of clinical presentations, from subtle
% sensory disturbances or brief lapses in awareness to overt, dramatic convulsive episodes, means that seizures can often be difficult to recognize or accurately describe,
% particularly if unwitnessed.

The diagnostic process for seizures heavily relies on meticulous clinical history-taking, including detailed eyewitness accounts, as the ictal events themselves are often
brief and unpredictable, making direct observation by clinicians rare~\cite{smith2015diagnosis}. This dependence on subjective reporting, while indispensable, introduces
potential inaccuracies and challenges in distinguishing seizures from other paroxysmal events~\cite{lafrance2008psychogenic}. Accurate seizure detection is not merely an
academic exercise; it is critical for guiding appropriate therapeutic interventions, predicting long-term outcomes, and informing genetic counseling~\cite{devinsky2018epilepsy}.

\section{Electroencephalography for Brain Activity Monitoring}

Electroencephalography (EEG) is a cornerstone neurophysiological technique used to record the electrical activity generated by the brain. It provides a non-invasive, direct
measure of brain function by detecting voltage fluctuations resulting from ionic current flows within the neurons of the cerebral cortex~\cite{niedermeyer2005electroencephalography,
kandel2013principles}. Specifically, EEG signals primarily reflect the summed postsynaptic potentials (both excitatory and inhibitory) of large populations of synchronously
active pyramidal neurons oriented radially to the scalp~\cite{kandel2013principles, sanei2007eeg}. Due to its excellent temporal resolution, typically in the millisecond
range, EEG is uniquely suited for capturing the rapidly changing dynamics of brain activity, making it an invaluable tool in both clinical neurology and neuroscience
research, particularly for the assessment of conditions characterized by abnormal electrical discharges, such as epilepsy~\cite{noachtar1999semiology}.

The acquisition of EEG data involves placing electrodes on the scalp, typically made of conductive materials like silver/silver-chloride (Ag/AgCl). These electrodes detect
the minute electrical potentials (on the order of microvolts, µV) generated by the brain, which are then significantly amplified by differential amplifiers to make them
suitable for digitization and subsequent analysis~\cite{teplan2002fundamentals}. To ensure reproducibility and comparability of EEG recordings across different laboratories
and individuals, standardized electrode placement systems are employed. The most widely adopted is the International 10-20 system and its extensions~\cite{klem1999ten,
oostenveld2001guidelines}. This system positions electrodes at locations that are 10\% or 20\% of the total front-to-back or right-to-left distance of the skull, ensuring
proportional spacing relative to cranial landmarks (nasion, inion, and preauricular points). Each electrode is labeled with a letter indicating the underlying brain lobe
(e.g., F for frontal, P for parietal, T for temporal, O for occipital) and a number or another letter to denote its specific position~\cite{klem1999ten}.

The resulting EEG waveforms are complex and are traditionally analyzed in terms of their frequency, amplitude, morphology, and topography. Clinically relevant information is
often contained within specific frequency bands, such as delta (0.5-4 Hz), theta (4-8 Hz), alpha (8-13 Hz), beta (13-30 Hz), and gamma (>30 Hz), each associated with
different brain states or cognitive processes~\cite{kandel2013principles}. In the context of epilepsy, EEG is crucial for identifying interictal epileptiform discharges
(IEDs) and for characterizing the electrographic signature of seizures themselves (ictal patterns)~\cite{stlouis2009epilepsy}. These patterns provide critical information
for diagnosing epilepsy, classifying seizure types, localizing the seizure onset zone, and guiding treatment decisions.

However, the utility of EEG is often challenged by the inherent low signal-to-noise ratio (SNR) and its susceptibility to various artifacts~\cite{teplan2002fundamentals,
islam2018eeg}. EEG signals are frequently contaminated by physiological artifacts originating from non-cerebral sources, such as eye movements and blinks (electrooculogram,
EOG), muscle activity (electromyogram, EMG) particularly from scalp and facial muscles, and cardiac activity (electrocardiogram, ECG)\cite{islam2018eeg, jiang2019removal}.
Non-physiological artifacts also pose significant problems, including 50/60 Hz power line interference, electrode impedance issues (e.g., electrode pop or poor contact),
and movement artifacts from patient motion or cable sway\cite{jiang2019removal}. These artifacts can mimic or obscure true epileptiform activity, leading to potential
misinterpretations if not properly addressed. Consequently, rigorous data preprocessing, encompassing techniques such as filtering, artifact detection, and artifact removal
or suppression, is an indispensable yet often complex step before any meaningful analysis or automated interpretation of EEG data can be performed~\cite{jiang2019removal,
fatourechi2007emg}. The challenge of reliably distinguishing pathological neural signals from this pervasive noise, especially in long-term recordings or in ambulatory
settings, underscores the need for robust and sophisticated signal processing and machine learning algorithms, forming a critical motivation for the research presented in
this thesis.

\section{Time-series classification}

Time Series Classification (TSC) is a specialized area within machine learning concerned with assigning predefined categorical labels to unlabeled time series
data~\cite{bagnall2017great}. A time series itself is a sequence of data points indexed in time order, commonly encountered in diverse domains such as finance, healthcare,
environmental science, and industrial processes~\cite{xing2010brief}. The primary objective of TSC is to build a model that can learn discriminative patterns or features
from a collection of labeled time series, enabling it to accurately predict the class of new, unseen time series instances, and providing a powerful framework for
automated detection tasks~\cite{bagnall2017great, esling2012time}.

Early TSC methods often relied on distance-based measures, with Dynamic Time Warping (DTW) being a prominent example, which calculates similarity between two temporal
sequences that may vary in time or speed~\cite{rakthanmanon2013searching}. Other approaches focused on extracting statistical or structural features from the time series
(e.g., mean, variance, spectral properties, shapelets) and then applying standard static classification algorithms~\cite{esling2012time, ye2009time}. While effective in
certain contexts, these methods can sometimes struggle with the high dimensionality, inherent noise, and complex temporal dependencies often present in real-world time
series data.

More recently, deep learning architectures have demonstrated state-of-the-art performance on many TSC benchmarks~\cite{fawaz2019deep}. Prominent among these are variants
of Convolutional Neural Networks (CNNs) and Residual Networks (ResNet), adapted for time series, and have achieved remarkable success by effectively capturing local and
global patterns~\cite{wang2017time}. Methods like InceptionTime also utilizes modules with multiple convolutional filter sizes to capture features at different temporal
scales~\cite{fawaz2019inceptiontime}. Recurrent Neural Networks (RNNs), particularly Long Short-Term Memory (LSTM) units and Gated Recurrent Units (GRUs), are inherently
suited for modeling temporal dependencies and have also been widely applied, though sometimes at a higher computational cost~\cite{ismail2019deep}. Furthermore,
Transformer-based models, leveraging self-attention mechanisms, have shown significant promise in capturing long-range dependencies and complex relationships within time
series data~\cite{zerveas2021transformer}. Other approaches also includes ensemble methods, which combine several heterogeneous classifiers into a large
meta-ensemble~\cite{bagnall2015cote, lines2018hivecote, middlehurst2021hivecote2}. However, training these models can be computationally intensive, require large datasets,
which poses limits in a sparse data scenario.

% (Advances in feature: catch22, hctsa, ROCKET and its variants, Detach-ROCKET)

Alongside the advancements in deep learning, a distinct and highly competitive paradigm has gained prominence, focusing on transform-based feature engineering coupled with
efficient classification. This approach decouples the often complex task of feature extraction from the subsequent classification step, allowing for specialized, powerful
feature generation techniques to be paired with simpler, faster classifiers. Early and comprehensive exemplars of this philosophy include toolkits like hctsa (Highly
Comparative Time-Series Analysis)\cite{fulcher2013highly} and catch22\cite{lubba2019catch22}. hctsa facilitates the extraction of thousands of diverse time-series features
derived from a wide array of scientific literature, enabling extensive data-driven exploration and the potential discovery of novel discriminative patterns~\cite{fulcher2013highly}.
In contrast, catch22 offers a more concise, curated set of 22 canonical time-series characteristics selected for their broad applicability and high discriminative power in
general-purpose TSC tasks, aiming for robust performance with minimal redundancy~\cite{lubba2019catch22}. Both these methodologies transform raw time series into a rich
feature space, thereby empowering relatively simple classifiers to perform effectively.

Another highly influential branch within this ``transform-then-classify" paradigm emerged with ROCKET (RandOm Convolutional KErnel Transform)\cite{Dempster2020rocket}.
ROCKET demonstrated that convolving time series with a large number of randomly generated, diverse convolutional kernels and extracting simple summary statistics (typically
the maximum value and proportion of positive values) could yield highly discriminative features. These features then enable robust linear classifiers, such as ridge
regression, to achieve state-of-the-art accuracy with exceptional speed, bypassing the need for complex backpropagation and extensive hyperparameter tuning associated
with many deep learning models\cite{Dempster2020rocket, ruiz2021greatmv}. The success of ROCKET spurred further innovations: MiniROCKET significantly reduced computational
overhead by optimizing kernel parameters and focusing primarily on the proportion of positive values feature~\cite{Dempster2021mini}, while MultiROCKET enhanced performance
by incorporating a richer set of summary statistics from the convolved outputs and an improved kernel generation strategy~\cite{tan2022multirocket}. Building upon this
lineage, Detach-ROCKET further reinforces the modular nature of the approach by separating, or "detaching" the redundant random kernels from the classifier training
stage~\cite{uribarri2024detachrocketsequentialfeatureselection}. This design preserves computational efficiency and enhances the flexibility of feature generation, thereby
readily supporting ensemble methods such as Detach Ensemble~\cite{solana2024classificationrawmegeegdata}, where diverse feature sets can be strategically combined. The
collective strength of these methods lies in their computational efficiency, reduced reliance on massive datasets, and potential for greater interpretability, making them
particularly well-suited for challenging biomedical signal analysis tasks.

\section{Related works}

The automated detection of epileptic seizures from EEG signals has been an active area of research for several decades, driven by the need to alleviate the laborious and
subjective process of manual EEG review by neurologists, and enable timely clinical intervention~\cite{roy2019review, usmankhujaev2021review}.

Early approaches predominantly relied on extracting salient features from EEG segments in various domains: time, frequency, and time-frequency—followed by classification
using conventional machine learning algorithms \cite{acharya2013automated_seizure_detection, hussein2020epileptic}. Common features include statistical measures (e.g.,
variance, kurtosis), spectral power in different EEG bands (delta, theta, alpha, beta, gamma), wavelet coefficients, and entropy measures \cite{gupta2021review,
kiral2023machine_learning_wavelet_emd}. Classifiers such as Support Vector Machines (SVMs), k-Nearest Neighbors (k-NN), Random Forests (RF), and Artificial Neural Networks
(ANNs) were then trained on these extracted features to distinguish ictal (seizure) from interictal (non-seizure) or normal EEG patterns \cite{subasi2019practical_guide,
kiral2023machine_learning_wavelet_emd}. While these methods have shown success, their performance heavily depends on the quality and relevance of the handcrafted features,
which often require domain expertise and may not generalize well across diverse patient populations or recording conditions \cite{roy2019review, usmankhujaev2021review}.

More recently, deep learning techniques have demonstrated the ability to automatically learn hierarchical feature representations directly from raw or minimally processed
EEG data \cite{shoeibi2021epileptic_seizures_deep_learning_review, anwar2022deep_learning_eeg_review, eldele2021attention_based_sleep_stage_eeg_transformer}. Convolutional
Neural Networks (CNNs) have been widely adopted for their proficiency in capturing spatial and temporal patterns from EEG signals, often treating multichannel EEG as an
image or a set of parallel time series \cite{acharya2018deep_cnn_seizure_detection, zhou2018patient_specific_cnn_fft, golmohammadi2019deep_learning_eeg_review}. Recurrent
Neural Networks (RNNs), particularly LSTMs and GRUs, are well-suited for modeling the temporal dynamics and long-range dependencies in EEG sequences \cite{hussein2019human_interpretable_lstm_seizure,
tsiouris2018long_short_term_memory_seizure_detection}. Hybrid models combining CNNs for feature extraction and RNNs for sequence modeling (CNN-LSTM) have also shown
promising results \cite{aslam2022cnn_lstm_seizure_prediction, indurani2023time_attention_cnn_lstm}. Furthermore, Transformer models, with their self-attention mechanisms,
are increasingly being explored for EEG analysis, including seizure detection, due to their capability to capture global dependencies in long sequences \cite{eldele2021attention_based_sleep_stage_eeg_transformer,
zhao2023automated_transformer_seizure_detection, tian2021hybrid_cnn_transformer_seizure_detection, wang2022transformer_seizure_prediction_graph, koshiyama2023unsupervised_transformer_seizure_identification,
gerster2023tsd_transformers_seizure_detection, liu2024automatic_s_transform_transformer}. These deep learning approaches often achieve good performance but typically require
substantial amounts of labeled data for training and can be computationally intensive \cite{shoeibi2021epileptic_seizures_deep_learning_review, anwar2022deep_learning_eeg_review}.

Some recent works also explore the application of ROCKET-based methodologies for EEG feature learning in seizure detection, aiming to combine the speed of random convolutional
kernel transforms with effective classification \cite{zhang2024efficient_eeg_rocket_wavelet, lundy2021rocket_neonatal_eeg}. These approaches seek to provide a balance
between high accuracy and computational efficiency, which is crucial for practical clinical applications, especially in resource-constrained environments or for real-time
monitoring systems \cite{lee2022real_time_seizure_detection_comparison, saleh2023tiny_ml_seizure_detection}.

\cleardoublepage
\chapter{Method}
\label{ch:methods}

\section{Data Processing}
\subsection{Data}

In this project, two datasets are employed for model training and evaluation. The first dataset is the TUH EEG Seizure Corpus~\cite{Shah2018TUH} (TUSZ), which comprises
recordings from a total of 459 subjects. Within this corpus, a subset of the EEG recordings contains annotated seizure events, while the remainder consists of background
(BCKG) recordings included to balance the dataset and to better assess the system's false alarm performance. Here BCKG recordings are defined as those that do not exhibit
any of the following patterns: spike and/or sharp waves, periodic lateralized epileptiform discharges, generalized periodic epileptiform discharges, eye movements, or
artifacts. Each EEG recording consists of 19 channels corresponding to the standard 10–20 system scalp electrode montage (see Figure~\ref{fig:electrodes}), with a sampling
rate of 256 Hz. See the publication of Obeid et al.~\cite{obeid2018temple} for further details on data collection.

Another dataset we used is the Siena Scalp EEG Database~\cite{Detti2020Siena}. This database contains EEG recordings from 14 patients, collected at the Unit of Neurology
and Neurophysiology at the University of Siena. The participants include 9 males (ages 25–71) and 5 females (ages 20–58). Each subject was monitored using Video-EEG at
a sampling rate of 256 Hz. The recordings include 19 EEG channels, based on the standard 10–20 system scalp electrode montage (see Figure~\ref{fig:electrodes}). See the
publication of Detti et al.~\cite{Detti2020} for further details on data collection.

\begin{figure}[!ht]
\begin{center}
\includegraphics[width=0.5\textwidth]{figures/electrodes.png}
\end{center}
\caption{Electrode locations for EEG signals. From Shah, V. et al~\cite{Shah2018TUH}}
\label{fig:electrodes}
\end{figure}

\subsection{Data preprocessing}
\label{method: datapreprocess}

To support model training across multiple datasets, we convert each dataset to the Seizure Community Open-Source Research Evaluation (SzCORE) standardized format~\cite{szcore}
for data and seizure annotations. This format is compatible with the Brain Imaging Data Structure (BIDS).

The model is trained on the large Temple University Seizure Corpus (TUSZ) dataset, which includes 1,134 recordings with seizures and 4,585 background recordings. All
recordings are downsampled to 128 Hz. Each recording is segmented into non-overlapping 10-second epochs. These epochs are then categorized as seizure, interictal, or
background (bckg). Seizure epochs are sampled from periods annotated as seizures. Interictal epochs are sampled from intervals between annotated seizure events. Background
epochs are sampled exclusively from recordings labeled as BCKG. Seizure epochs are assigned the label 1 (seizure), while interictal and background epochs are assigned
the label 0 (non-seizure).

\section{Model} \todo{Do you have any text to describe your own adaptations of this to make it run? You want to help a student to be able to replicate your work by showing what work you needed to do to make this into a computational pipeline.}
\subsection{ROCKET and MINIROCKET}

ROCKET (\textbf{R}and\textbf{O}m \textbf{C}onvolutional \textbf{KE}rnel \textbf{T}ransform) achieves state-of-the-art classification accuracy for time series classification
while requiring only a fraction of the computational resources used by most existing methods~\cite{Dempster2020rocket}. It transforms time series using random convolutional
kernels and uses the transformed features to train a linear classifier. The method was later reformulated as MINIROCKET (\textbf{MINI}mally \textbf{R}and\textbf{O}m
\textbf{C}onvolutional \textbf{KE}rnel \textbf{T}ransform), which is up to 75 times faster than ROCKET on large datasets and is almost deterministic, while maintaining
nearly the same accuracy~\cite{Dempster2021mini}. MINIROCKET is significantly faster than other methods with similar accuracy and provides significantly better accuracy
than methods with comparable computational efficiency.

\subsection{Detach-ROCKET}
\label{method:detachrocket}
Although ROCKET and MINIROCKET are efficient and computationally lightweight, many of the randomly generated features are redundant or non-informative, increasing the
computational burden and may reduce the model's generalizability. Detach-ROCKET addresses this issue by introducing Sequential Feature Detachment (SFD), a method designed
to identify and prune the non-essential features from ROCKET-based models~\cite{uribarri2024detachrocketsequentialfeatureselection}. In SFD, the transformed features
are ranked according to their contribution to the model's decisions. At each iteration, a fixed proportion of the least informative features is discarded. Let $\mathbb{F}$
denote the complete set of features generated by ROCKET's random convolutional kernels. The subset of currently active features is represented by $\mathbb{S}_t$, which
is initially set to $\mathbb{F}$.

At each step $t$, a ridge classifier is trained on the active feature set $\mathbb{S}_t$ by solving the following optimization problem:

\begin{equation}
\begin{aligned} {\hat{\theta \ }_t}^{\textrm{ridge}}=\underset{\theta }{{\text {argmin}}}\left\{ \sum _{i=1}^N\left( y_i-\theta _0-\sum _{k \in \varvec{\mathbb {S}}_t} x_{i k} \theta _k\right) ^2+\lambda \sum _{k \in \varvec{\mathbb {S}}_t} \theta _k^2\right\} \end{aligned}
\end{equation}

The training process produces a set of optimal coefficients $\hat{\theta}_t^{\text{ridge}}=\{\hat{\theta}_k\}$, where each coefficient is proportional to the contribution
of a corresponding feature on the classifier's decision. The features are then ranked according to the absolute values of their coefficients. Let $p$ denote the proportion
of features to be removed. At each step, the lowest $100 \cdot p\%$ of ranked features discarded, and the remaining $100 \cdot (1-p)\%$ are retained to form the updated
feature set $\mathbb{S}_{t+1}$. Since $p$ controls the trade-off between computational cost and model accuracy, it is set to 0.05 to ensure a conservative yet computationally
affordable pruning procedure, meaning that $5\%$ of the features are removed at each step.

To determine the optimal number of features to finally retain in our dataset, we solve the following optimization problem:

\begin{equation}
\begin{aligned} Q_c = \underset{q}{\text{argmax}} \ f_c(q) = \underset{q}{\text{argmax}} \ \left\{ \alpha(q)+c \cdot q \right\} \end{aligned}
\end{equation}

In this equation, $Q_c$ represents the accuracy curve obtained by evaluating the model at each pruning step, $q$ represents the proportion of pruned features, and $\alpha(q)$
the accuracy of the pruned model on the validation set. The hyperparameter $c$ is the weighting factor between accuracy and data size, in which a smaller $c$ value favors
accuracy and larger $c$ value favors data size. We set $c=0.1$, as this achieves the optimal performance according to Fig.4 in ~\cite{uribarri2024detachrocketsequentialfeatureselection}.

\subsection{Detach Ensemble}

Given the high dimensionality of multivariate time series such as the 19-channel EEG data in this project, relying on a single set of randomly generated kernels may
inadequately capture the complex spatio-temporal patterns and inter-channel relationships critical for seizure detection. To address this limitation and improve model
generalization and robustness, an ensemble approach based on Detach-ROCKET is employed. The Detach Ensemble involves training $N$ independent Detach-ROCKET models. For
each model, a subset of the training data is used to determine the optimal pruning size, and the model is then pruned using SFD. Each pruned model is then assigned a
weight based on its performance on the training set~\cite{solana2024classificationrawmegeegdata}.

For classifying a given input instance, predictions from each individual model are aggregated via a weighted average, using the predetermined model weights, to yield a
final ensemble probability score. This probability is then thresholded (commonly at 0.5) to produce the definitive classification label (seizure or non-seizure).

In our implementation, we use an ensemble of $N=10$ Detach-MiniROCKET models. This number was chosen as a balance between seeking improved performance through ensemble
diversity and managing computational resources, as training significantly more models becomes resource-intensive. Each of the 10 model is trained independently using the
procedure described in Section ~\ref{method:detachrocket}, including the application of SFD with $p=0.05$ and the feature selection criterion $c=0.1$.

\section{Evaluation of Models}

The performance of the trained Detach-ROCKET and Detach Ensemble model are assessed through two complementary evaluation frameworks: epoch-wise and event-wise analysis.

Firstly, an epoch-wise evaluation is conducted based on the 10-second non-overlapping segments described in Section~\ref{method: datapreprocess}. Standard classification
metrics including accuracy, sensitivity, precision, and the F1-score are computed on the test set. The confusion matrix is also examined to understand the distribution
of classification errors. This epoch-based assessment provides views of the model's ability to correctly classify individual epochs.

However, recognizing that clinical assessment of epilepsy monitoring often focuses on the detection of seizure episodes (events) rather than isolated time epochs, an
event-wise evaluation is performed. During the evaluation procedure, event-based sensitivity, precision, and F1-score are calculated base on the open-source \texttt{timescoring}
library~\cite{timescoring2023}.

\subsection{Evaluation metrics}

To evaluate the performance both epoch-wise and event-wise, we first define and explain the True Positives (TP), True Negatives (TN), False Positives (FP), and False
Negatives (FN) in this study: True Positive (TP): An instance correctly identified as belonging to the positive class (seizure); True Negative (TN): An instance correctly
identified as belonging to the negative class (non-seizure);False Positive (FP): An instance incorrectly identified as belonging to the positive class when it belongs
to the negative class (a non-seizure instance classified as seizure);False Negative (FN): An instance incorrectly identified as belonging to the negative class when it
belongs to the positive class (a seizure instance classified as non-seizure).

For epoch-wise evaluation, a TP is a correctly classified seizure epoch, an FP is a non-seizure epoch classified as seizure, etc. For event-wise evaluation, the following
parameters were configured to combine the epochs to seizure events, as suggested by the 2025 Seizure Detection Challenge:

\begin{itemize}
\item Minimum Overlap: Any temporal overlap, however brief, between a reference event and a hypothesis event is sufficient to consider it a potential match. This setting
maximizes sensitivity to detecting any part of a seizure.
\item Pre-ictal Tolerance: A hypothesis event starting up to 30 seconds before the onset of a reference event can still be considered a detection of that event.
\item Post-ictal Tolerance: A hypothesis event ending up to 60 seconds after the end of a reference event can still be considered part of the detection of that event.
\item Minimum Duration: Reference or hypothesis events separated by less than 90 seconds are merged into a single, longer event before scoring. This duration corresponds
to the sum of the pre- and post-ictal tolerances, preventing closely spaced detections from being penalized multiple times.
\end{itemize}

In this configuration, a reference event refers to a labeled seizure event, a hypothesis event refers to a model predicted event. Based on these criteria, an event-based
TP occurs when a reference seizure event is correctly matched with one or more hypothesis seizure events according to the overlap and tolerance rules. An event-based FP
corresponds to a hypothesis seizure event that does not match any reference event. An event-based FN represents a reference seizure event that is not matched by any
hypothesis event. For the specific application of seizure detection, event-wise metrics are considered more clinically meaningful than epoch-wise metrics. Consequently,
while both will be reported, event-wise performance will be the primary focus for evaluating the model's practical utility.

Based on these setting, we define the accuracy, sensitivity, precision, f1-score and false alarm rate as following:

\begin{itemize}
\item Accuracy: the proportion of total instances that were correctly classified.
\begin{equation}
\text{Accuracy} = \frac{TP + TN}{TP + TN + FP + FN}
\end{equation}
\item Sensitivity: the proportion of true positive instances that were correctly identified by the model.
\begin{equation}
\text{Sensitivity} = \frac{TP}{TP + FN}
\end{equation}
\item Precision: the proportion of instances predicted as positive that were actually positive.
\begin{equation}
\text{Precision} = \frac{TP}{TP + FP}
\end{equation}
\item F1 Score: the harmonic mean of Precision and Sensitivity, providing a single metric that balances both concerns.
\begin{equation}
\text{F1 Score} = 2 \times \frac{\text{Precision} \times \text{Sensitivity}}{\text{Precision} + \text{Sensitivity}} = \frac{2 \times TP}{2 \times TP + FP + FN}
\end{equation}
\item False positive rate (FPR): the number of false positives per 24 hours. A low FPR is crucial for clinical usability, as frequent false alarms can reduce clinician
trust in the system.
\begin{equation}
\text{FPR} = \frac{FP}{FP + TN}
\label{eq:fprate}
\end{equation}
\end{itemize}

\section{Channel Relevance Estimation}

To estimate the relevance of individual EEG channels within our Detach-ROCKET Ensemble, we adopted the methodology proposed by Solana et al. ~\cite{solana2024classificationrawmegeegdata}.
This process first assesses channel relevance on each Detach-MINIROCKET model:

\begin{enumerate}
\item \textbf{Kernel Selection:} Sequential Feature Detachment (SFD) identifies and selects the most relevant kernels within the MiniROCKET model.
\item \textbf{Channel Retrieval:} For each kernel selected by SFD, the specific EEG channels it processed are retrieved.
\item \textbf{Importance Weighting:} Each retrieved channel receives an importance score proportional to its kernel's weight ($\theta_i$). It is then divided by the number
of channels in that same kernel.
\item \textbf{Relevance Aggregation:} The weighted importance scores for all channels are summed across the selected kernels. These sums are then normalized to create
a relative channel relevance histogram.
\end{enumerate}

The final ensemble relevance for each channel is then determined by taking the median of relevancies from base models and normalizing across all channels.

\cleardoublepage

\chapter{Results and Analysis}
\label{ch:resultsAndAnalysis}

\section{Experiments on TUSZ Corpus}
\subsection{Training and test with Detach-MINIROCKET and Ensemble}

In this project, we trained two models, Detach-ROCKET and Detach Ensemble, on the TUSZ dataset.

Due to hardware constraints and the large size of the dataset, we randomly select 50\% of the available subjects to construct the working dataset. This subset is then
divided into training and test sets based on subject identifiers to ensure that the evaluation reflects the model's generalizability across individuals. We allocate 80\%
of the subjects to the training set and the remaining 20\% to the test set. Each subject includes multiple sessions and runs, with each run containing several recordings
labeled as either seizure or background (BCKG). See \todo{add plot of data split}. Notably, a portion of the subjects in the dataset contribute exclusively BCKG recordings,
without any associated seizure activity.

To accommodate memory limitations and promote a balanced and diverse training set, after segmenting the recordings in the training set into non-overlapping 10-second
epochs, we include all available seizure epochs, randomly select an equal number of interictal epochs, and randomly sample half that number of background epochs for model
training. For the evaluation of epoch-wise accuracy, this methodology resulted in an average of approximately 20,000 epochs being used for model training, with an average
of about 5,000 epochs allocated for testing. \todo{add confusion matrix?} The models' performances were initially assessed using epoch-wise accuracy. Table~\ref{tab:modelaccuracy}
shows the accuracy of both models, averaged over three trials.

% As the training and test sets were partitioned at the subject level, these results indicate the model's capacity to generalize effectively across individual variations.

The models are then evaluated on a dataset comprising the same subjects included in the test set during model training, while using all the epochs produced by these
subjects without selecting a subset of them. Note that the evaluation dataset, though comprising the same subjects as test set, is however unbalanced with more non-seizure
epochs than seizure epochs. Performance was evaluated using sensitivity, precision, F1-score, and False Positive Rate (FPR), computed at the recording level for both
epoch-wise and event-wise analyses. Table~\ref{tab:standardmodelperformance} shows the performance metrics for the single Detach-MINIROCKET model using 10,000 features
and 10-model Detach-MINIROCKET ensemble, each of the base models using 10,000 features. These reported values are averaged first across all evaluation recordings and
then across the three random data partitions. The high event-wise F1-score of both models demonstrates the model's strong generalization capabilities across different
individuals. Figure ~\ref{fig:example_evaluation} provides examples of the models' output on representative recordings containing either seizure or background EEG patterns.

\begin{figure}[htbp]
\centering
\begin{subfigure}[b]{0.48\textwidth}
\centering
\includegraphics[width=\linewidth]{figures/TUSZ_sub-203_ses-03_szMonitoring_run-04.png}
\caption{Correctly predicts all seizure events (TP) with no FP.}
\label{fig:subfig_TP}
\end{subfigure}
\hfill % Adds horizontal flexible space between subfigures on the same line
\begin{subfigure}[b]{0.48\textwidth}
\centering
\includegraphics[width=\linewidth]{figures/TUSZ_sub-239_ses-00_szMonitoring_run-01.png} % Replace example-image-b with your_image_file2.png/jpg/pdf
\caption{Fails to predict an existing seizure event (FN).}
\label{fig:subfig_FN}
\end{subfigure}

\vspace{0.5cm} % Optional: Adds a small vertical space between rows

% Second row of subfigures
\begin{subfigure}[b]{0.48\textwidth}
    \centering
    \includegraphics[width=\linewidth]{figures/TUSZ_sub-193_ses-09_szMonitoring_run-09.png} % Replace example-image-c with your_image_file3.png/jpg/pdf
    \caption{Incorrectly predicts a seizure when no seizure is present (FP).}
    \label{fig:subfig_FP}
\end{subfigure}
\hfill % Adds horizontal flexible space
\begin{subfigure}[b]{0.48\textwidth}
    \centering
    \includegraphics[width=\linewidth]{figures/TUSZ_sub-193_ses-19_szMonitoring_run-01.png} % Replace example-image-d with your_image_file4.png/jpg/pdf
    \caption{Correctly predicts the absence of seizure (TN).}
    \label{fig:subfig_TN}
\end{subfigure}

\caption{Examples of model evaluation outcomes.}
\label{fig:example_evaluation}


\end{figure}

To further assess the model's generalization capabilities across individual subjects, subject-wise performance metrics were computed on the test set. Figure~\ref{fig:subjectscoretusz}
illustrates the distribution of event-wise F1 scores and False Positive Rates(FPR per 24h) across individual subjects from a representative evaluation, revealing considerable
inter-subject variability. For both models, a significant proportion of subjects obtained F1 scores in the highest bin of 0.9-1.0. FPR values also showed considerable
variation across subjects, with a significant proportion of subjects demonstrating low false alarm rates(below 30). Despite this observed inter-subject performance variation,
these distributions suggest that both models generalize effectively for a majority of individuals. It is important to note that the F1 score distribution presented in
Figure~\ref{fig:subjectscoretusz} excludes subjects who contributed exclusively background (BCKG) recordings to the test set (i.e., had no annotated seizure events). For
these subjects, True Positives (TP) and False Negatives (FN) are inherently zero, so the F1 score cannot be calculated.

\begin{figure}[!ht]
\centering
\begin{subfigure}[b]{0.48\textwidth}
\centering
\includegraphics[width=\linewidth]{figures/subjectf1_demini.png}
\caption{Subject-wise event F1 score distribution for the Detach-MINIROCKET model on the TUSZ test set.}
\label{fig:subf1_d}
\end{subfigure}
\hfill
\begin{subfigure}[b]{0.48\textwidth}
\centering
\includegraphics[width=\linewidth]{figures/subjectfpr_demini.png}
\caption{Subject-wise event FPR distribution for the Detach-MINIROCKET model on the TUSZ test set.}
\label{fig:subfpr_d}
\end{subfigure}

\vspace{0.5cm} 

% Second row of subfigures
\begin{subfigure}[b]{0.48\textwidth}
    \centering
    \includegraphics[width=\linewidth]{figures/subjectf1_tusz.png} 
    \caption{Subject-wise event F1 score distribution for the Detach-MINIROCKET Ensemble on the TUSZ test set.}
    \label{fig:subf1_e}
\end{subfigure}
\hfill 
\begin{subfigure}[b]{0.48\textwidth}
    \centering
    \includegraphics[width=\linewidth]{figures/subjectfpr_tusz.png} 
    \caption{Subject-wise event FPR distribution for the Detach-MINIROCKET Ensemble on the TUSZ test set.}
    \label{fig:subfpr_e}
\end{subfigure}

\caption{Distribution of subject-wise performance metrics on the TUSZ test set for different model configurations. }
\label{fig:subjectscoretusz}

\end{figure}

While both models demonstrate good event-wise F1 scores and generalization across subjects, the Detach-MINIROCKET model exhibits lower epoch-wise performance. This suggests
that, compared to the Detach Ensemble model, it's difficult for a single Detach-ROCKET model with 10,000 features to sufficiently sample the feature space and encode
meaningful multichannel information. Despite this limitation, the single Detach-MINIROCKET model is significantly more computationally efficient, as it requires training
only one model. On the other hand, although the Detach Ensemble model introduces higher computational costs, it offers enhanced model interpretability by providing a
built-in way to generate label probabilities and estimate channel relevance, which is particularly valuable in the context of seizure detection.

\subsection{Benchmark}
To establish a robust, interpretable, and accessible baseline for our models, we selected catch22~\cite{lubba2019catch22} as our benchmark. catch22 offers a concise set
of 22 highly discriminative and computationally inexpensive time-series features, distilled from the much larger hctsa feature set~\cite{fulcher2013highly}. Its design
emphasizes generalizability and interpretability across diverse time series domains \cite{bagnall2017great, lubba2019catch22}. Furthermore, the widespread availability
and easy implementation of catch22 contribute to its utility as a practical and reproducible benchmark for evaluating the performance and computational costs of more
complex or specialized algorithms in TSC tasks.

Our models and catch22 are evaluated on the TUSZ dataset. Table~\ref{tab:modelaccuracy} and Table~\ref{tab:standardmodelperformance} summarize a comparative performance
with the benchmark model, averaged over three distinct random splits, each using 50\% of the total available subjects. Our results indicate that the Detach Ensemble
model outperforms the other models in overall test accuracy, as well as in both epoch-wise and event-wise F1 scores. Specifically, Detach Ensemble achieved a test accuracy
approximately 10\% higher than Detach-MINIROCKET and 2\% higher than catch22. For event-wise F1 scores, Detach Ensemble outperforms catch22 by 4\% and Detach-MINIROCKET
by 1\%. For epoch-wise F1 scores, Detach Ensemble substantially outperforms both catch22 and Detach-MINIROCKET by 6\%. Notably, both Detach-MINIROCKET and the Detach
Ensemble achieved higher event-wise F1 scores than catch22, while requiring significantly lower computational resources, as detailed in Section~\ref{sec:efficiency}.
Furthermore, Detach-MINIROCKET exhibited the lowest False Positive Rate (FPR) among the evaluated models.

\begin{table}[!ht]
\begin{center}
\caption{Comparison of epoch-wise training and test accuracies for the single Detach-MINIROCKET (D-MINIROCKET) model, Detach-MINIROCKET Ensemble (D-MINIROCKET Ens.) and catch22.}\todo{mean+std(range?), or median}
\label{tab:modelaccuracy}
\begin{tabular}{l|S[table-format=.2]|S[table-format=.2]|S[table-format=.2]}
& \textbf{Train Acc.} & \textbf{Test Acc.} \\
\hline
D-MINIROCKET & 0.95 & 0.71  \\
D-MINIROCKET Ens. & 0.97 & 0.81  \\
catch22 & 1.00 & 0.79 \\
\end{tabular}
\end{center}
\end{table}

\begin{table}[!ht]
\begin{center}
\caption{Comparison of model performance for single Detach-MINIROCKET (D-MINI), 10-model Detach-MINIROCKET ensembles(D-MINI Ens.) and catch22, using 50\% of the dataset.}
\label{tab:standardmodelperformance}
\begin{tabular}{l|l % Column 1: Ensemble Type (multirow), Column 2: Evaluation Type
S[table-format=.2]   % Sensitivity
S[table-format=.2]   % Precision
S[table-format=.2]   % F1 score
S[table-format=5.0]}% <-- Alignments: 1st column left, 2nd middle and 3rd right, with vertical lines in between
\textbf{Config.} & \textbf{Level}& \textbf{Sens.} & \textbf{Prec.} & \textbf{F1} & \textbf{FPR}\\
\hline
\multirow{4}{}{D-MINI} & Epoch & 0.52 & 0.68 & 0.51 & 7915 \\
& \textit{Range} & \multicolumn{1}{c}{\textit{(0.50, 0.53)}} & \multicolumn{1}{c}{\textit{(0.66, 0.69)}} & \multicolumn{1}{c}{\textit{(0.49, 0.55)}} & \multicolumn{1}{c}{\textit{(6798, 9035)}} \\
& Event & 0.90 & 0.86 & 0.86 & 27 \\
& \textit{Range} & \multicolumn{1}{c}{\textit{(0.87, 0.91)}} & \multicolumn{1}{c}{\textit{(0.85, 0.86)}} & \multicolumn{1}{c}{\textit{(0.84, 0.87)}} & \multicolumn{1}{c}{\textit{(24, 30)}} \\
\hline
\multirow{4}{}{D-MINI Ens.} & Epoch & 0.72 & 0.60 & 0.57 & 14543 \\
& \textit{Range} & \multicolumn{1}{c}{\textit{(0.65, 0.76)}} & \multicolumn{1}{c}{\textit{(0.42, 0.72)}} & \multicolumn{1}{c}{\textit{(0.40, 0.67)}} & \multicolumn{1}{c}{\textit{(7999, 22843)}} \\
& Event & 0.95 & 0.85 & 0.87 & 37 \\
& \textit{Range} & \multicolumn{1}{c}{\textit{(0.91, 0.97)}} & \multicolumn{1}{c}{\textit{(0.78, 0.90)}} & \multicolumn{1}{c}{\textit{(0.81, 0.91)}} & \multicolumn{1}{c}{\textit{(22, 51)}} \\
\hline
\multirow{4}{*}{catch22} & Epoch & 0.60 & 0.63 & 0.51 & 9603 \\
& \textit{Range} & \multicolumn{1}{c}{\textit{(0.54, 0.65)}} & \multicolumn{1}{c}{\textit{(0.57, 0.71)}} & \multicolumn{1}{c}{\textit{(0.50, 0.52)}} & \multicolumn{1}{c}{\textit{(6022, 12103)}} \\
& Event & 0.88 & 0.84 & 0.83 & 36 \\
& \textit{Range} & \multicolumn{1}{c}{\textit{(0.84, 0.91)}} & \multicolumn{1}{c}{\textit{(0.81, 0.87)}} & \multicolumn{1}{c}{\textit{(0.82, 0.84)}} & \multicolumn{1}{c}{\textit{(32, 40)}} \\

\end{tabular}

\end{center}
\end{table}

\section{Sensitivity Analysis}

The performance of a single Detach-MINIROCKET model, and consequently its sensitivity to hyperparameter changes, can be significantly influenced by the stochastic nature
of its random convolutional kernel generation \cite{Dempster2020rocket, Dempster2021mini}. While this inherent randomness is integral to the efficiency of the ROCKET
family, it makes it more difficult to isolate the specific impact of individual hyperparameter adjustments in limited model instances, as it can have a large variation
due to randomness rather than solely the hyperparameter change. In contrast, Detach-Ensemble, by aggregating multiple Detach-MINIROCKET instances, reduces such stochastic
effects, leading to more stable and representative performance characteristics. Given the foundational homogeneity between Detach Ensemble and its constituent Detach-MINIROCKET
models (as the ensemble is built from them), our detailed sensitivity analysis was primarily focused on the Detach-Ensemble configuration.

% To comprehensively evaluate the models and understand the influence of its various components and hyperparameters, we studied model sensitivity to key parameter choices:
We therefore explored the effect of key parameter choices on the Detach-Ensemble's test accuracy, epoch-wise and event-wise F1 score. The key parameters investigated were:

\begin{enumerate}
\item number of models ($N$)
\item epoch duration
\item training data size
\end{enumerate}

Table~\ref{tab:hyperparam} presents the results for Detach Ensemble models configured with different $N$ and epoch durations trained on the same data split of the TUSZ
dataset. The results indicate that changing the epoch duration between 5 and 20 seconds has a relatively minor influence on event-wise F1 scores, ranging from 0.86 to
0.89 across these settings for an ensemble of $N=10$. However, the 5-second epoch model exhibits a substantially lower epoch-wise F1 score (0.45) compared to models
with 10-second (0.64) or 20-second (0.66) epochs. Furthermore, the 20-second epoch model demonstrates a notably lower test accuracy (0.68) when compared to the accuracies
achieved with 5-second (0.86) and 10-second (0.86) epoch models.

When keeping the epoch duration fixed at 10 seconds, increasing the number of models $N$ in the ensemble generally leads to improved performance, particularly in F1 scores.
Test accuracy rises from 0.73 at $N=5$ to 0.86 at $N=10$, but then slightly declines to 0.84 at $N=20$. This trend suggests that while larger ensembles improve the
model's ability to detect seizure epochs and events, there may be a point of diminishing or slightly negative returns for test accuracy. This may originate from the slight
overfitting to training set nuances or the inclusion of additional models that offer limited diversity relative to the existing models.

\begin{table}[!ht]
\begin{center}
\caption{Performance evaluation of the Detach-Ensemble model under varying hyperparameter configurations.}
\label{tab:hyperparam}
% \setlength{\tabcolsep}{3pt}
\begin{tabular}{S[table-format=2.0]|S[table-format=2.0]|S[table-format=.2]|S[table-format=.2]|S[table-format=.2]}
\textbf{$N$} & \textbf{Epoch Duration} & \textbf{Test Acc.} & \textbf{Epoch F1} & \textbf{Event F1} \\
\hline
10 & 10 & 0.86 & 0.64 & 0.89 \\
10 & 5 & 0.86 & 0.45 & 0.86 \\
10 & 20 & 0.68 & 0.66 & 0.89 \\
5 & 10 & 0.73 & 0.46 & 0.85 \\
20 & 10 & 0.84 & 0.69 & 0.92 \\

\end{tabular}

\end{center}
\end{table}

The impact of training data size on model performance is shown in Figure~\ref{fig:datasizemetrics}. A notable deterioration in performance is observed when the proportion
of training data used falls below 30\% of the total dataset. Beyond this point, the model achieves a relatively stable and high level of performance. This finding supports
our decision to use 50\% of the available data for training, as it offers a conservative, resource-efficient approach while ensuring robust model performance. The result
also suggests model's ability for effective generalization even when trained on limited dataset sizes.

\begin{figure}[!ht]
\centering
\includegraphics[width=0.8\textwidth]{figures/datasizemetrics.png}
\caption{Performance of the Detach-Ensemble model on varying training datasize proportions}
\label{fig:datasizemetrics}
\end{figure}

\section{Cross-Dataset Evaluation}

To evaluate the models' generalization capabilities on unseen data, we assess model performance on the Siena Scalp EEG Database. This independent dataset provides an
opportunity to test the model's robustness to variations in patient populations, recording equipment, and potential differences in EEG characteristics. Table~\ref{tab:crossdata}
presents the averaged performance of our models when evaluated on the entire Siena dataset, compared with catch22. The results demonstrate that both Detach-MINIROCKET
and Detach-Ensemble models achieve better F1 scores (both epoch-wise and event-wise) and lower event-wise False Positive Rates (FPR) compared to catch22 on this cross-dataset
validation task.

To further understand models' cross-dataset generalization capabilities, we refer to the publicly reported outcomes of the Seizure Detection Challenge 2025.

In this challenge, proposed algorithms were trained on publicly available datasets and then evaluated on a private EEG dataset (Dianalund Scalp EEG dataset), with rankings
primarily determined by their event-wise F1 score. Adhering to the same epoch-to-event aggregation rules employed in the challenge, our Detach-MINIROCKET and Detach-Ensemble
model achieved an event-wise F1 score of 0.41 and 0.42 on the Siena dataset, respectively. This performance is highly competitive with the top-ranked algorithms in the
challenge, which reported event-wise F1 scores of 0.43, 0.36, and 0.34. Given that the Siena dataset represents an entirely different data distribution from the training
set (TUSZ), the performance of our models indicates a strong capacity for generalization and robust seizure detection across diverse datasets.

\begin{table}[!ht]
\begin{center}
\caption{Cross-dataset performance comparison of Detach-MINIROCKET, Detach-MINIROCKET Ensemble, and catch22 on the Siena dataset.}
\label{tab:crossdata}
% \setlength{\tabcolsep}{3pt}
\begin{tabular}{l|S[table-format=.2]|S[table-format=.2]|S[table-format=4.0]}% <-- Alignments: 1st column left, 2nd middle and 3rd right, with vertical lines in between
& \textbf{Epoch F1} & \textbf{Event F1} & \textbf{Event FPR} \\
\hline
D-ROCKET & 0.36 & 0.41 & 55 \\
D-ROCKET Ens. & 0.34 & 0.42 & 58 \\
catch22 & 0.28 & 0.37 & 82\\

\end{tabular}

\end{center}
\end{table}

\section{Channel Relevance}

To investigate the relative importance of different EEG channels for seizure detection within our framework, we employed the channel relevance estimation method proposed
by Solana et al. \cite{solana2024classificationrawmegeegdata}. This method was applied to each of the well-performing Detach-MINIROCKET Ensemble models developed during
our training phase. Figure~\ref{fig:cr} illustrates the mean and standard deviation of the estimated channel relevance values, aggregated across these selected ensemble
models.

Observations from Figure~\ref{fig:cr} indicate that channels Fp1 and C4 exhibit notably high mean relevance values compared to other channels. However, these channels
also display relatively large standard deviations. Conversely, channel F8 presents with a significantly low mean relevance value and a comparatively small standard deviation,
indicating its consistently minor role in seizure detection for our models on the TUSZ dataset.

Further analysis involved ranking channels from most to least relevant within each individual ensemble model. This ranking revealed that channels Fp1, C4, and O1 were
frequently identified among the top four most relevant channels (appearing 4 out of 7 times for each of these channels across the evaluated models). In contrast, channel
F8 was most frequently found among the top four least relevant channels (6 out of 7 model instances), with channel T3 also appearing frequently in this less relevant
group (4 out of 7 model instances).

The consistent findings combining these two analytical approaches suggest that channels Fp1 and C4 are likely more critical for our seizure detection task on the TUSZ
dataset, while channel F8 appears to be consistently less informative. In practical scenarios where computational resources are constrained, or where feature selection
is desired to emphasize inter-channel relationships or reduce model complexity, these findings suggest that prioritizing channels such as Fp1, C4, and potentially O1,
while deprioritizing F8 and possibly T3, could be a viable strategy.

\begin{figure}[!ht]
\centering
\includegraphics[width=0.8\textwidth]{figures/cr_std.png}
\caption{Mean and standard deviation of estimated channel relevance values across selected Detach-MINIROCKET Ensemble models. Higher mean values suggest greater overall importance, while standard deviation indicates variability across models.}
\label{fig:cr}
\end{figure}

\section{Model Efficiency}
\label{sec:efficiency}

The computational efficiency of the models, particularly their prediction speed, is a critical factor for practical deployment, especially in real-time seizure detection
scenarios. To evaluate this, prediction times were measured on an Intel(R) Xeon(R) Platinum 8480C processor for a representative EEG recording of 1208 seconds in duration.
The results are summarized in Table~\ref{tab:model_predict_time}.

As indicated in Table~\ref{tab:model_predict_time}, the single Detach-MINIROCKET model processed the entire recording in just 0.74 seconds, which is approximately 17.1
times faster than the catch22 benchmark. The Detach-MINIROCKET Ensemble, while inherently introducing a higher computational cost due to the aggregation of multiple individual
models, still completed the prediction in 6.88 seconds. This remains notably faster than catch22, by a factor of approximately 1.8. These findings highlight the significant
computational advantages of the ROCKET-based feature transformation approach employed by our models.In particular, the rapid inference capability of the single Detach-MINIROCKET
model demonstrates strong potential for applications requiring near real-time processing and decision-making. While the ensemble model requires increased prediction time,
it provides enhanced predictive performance and is therefore still a competitive choice.

\begin{table}[!ht]
\begin{center}
\caption{Model prediction time (seconds) on a 1208s EEG recording.}
\label{tab:model_predict_time}
% \setlength{\tabcolsep}{3pt}
\begin{tabular}{S[table-format=.2]|S[table-format=.2]|S[table-format=.2]}% <-- Alignments: 1st column left, 2nd middle and 3rd right, with vertical lines in between
\textbf{D-MINIROCKET} & \textbf{D-MINIROCKET Ens.} & \textbf{Catch22} \\
\hline
0.74s & 6.88s & 12.68s  \\

\end{tabular}

\end{center}
\end{table}

\cleardoublepage
\chapter{Discussion}
\label{ch:discussion}
\section{Benchmark selection}
We chose catch22 as our benchmark because it provides a general and universal solution for time series classification (TSC) tasks. As a state-of-the-art framework in
TSC, catch22 shares conceptual similarities with detach-ROCKET in its approach to feature pruning. The benchmark is easily accessible, well-documented, and has demonstrated
effectiveness across numerous TSC applications. Notably, EEG datasets are included in the calibration set of time series classification tasks that catch22 employs, which
provides some relevance to our seizure detection domain. However, it is important to acknowledge that catch22 is not specifically designed as a seizure detection method,
nor is it tailored for the unique characteristics and requirements of seizure detection tasks. This limitation indicates that direct comparisons may not constitute entirely
fair evaluations, as the benchmark was not optimized for the specific challenges inherent in seizure detection, such as the temporal dynamics and the importance of minimizing
false negatives in clinical applications.

It is important to point out that our model, Detach-ROCKET and Detach Ensemble, are not adjusted either for specific challenges in seizure detection. From this perspective,
it is reasonable to conduct such a comparison. Importantly, the computational efficiency of catch22 features serves as a baseline for understanding the relative performance
of our models in terms of efficiency. It also helps establish whether the additional complexity of our ensemble model yields meaningful improvements while not sacrificing
efficiency excessively.

\section{Subject Variations}
\label{sec:subjectvariations}

A key observation from our evaluation is the substantial inter-subject variability in model performance, as illustrated by the subject-wise scores on the TUSZ dataset
(Figure~\ref{fig:subjectscoretusz}). This heterogeneity is not unexpected and reflects the inherent complexities of EEG-based seizure detection. Several factors likely
contribute to these observed variations:

\subsection{Variations in Data Composition}

The composition of data available for each subject in the TUSZ corpus introduces a source of variability. Some subjects exclusively contribute background (BCKG) recordings,
meaning the model's performance for these individuals is solely based on its ability to correctly identify non-seizure states and avoid false positives. Also, the number
and length of seizures and the total duration of recordings can vary a lot between subjects with seizure events.

\subsection{Inherent Seizure Heterogeneity}
Epileptic seizures are highly heterogeneous phenomena, varying significantly not only between individuals but also potentially within the same individual over time. Subjects
may exhibit different seizure types (e.g., focal vs. generalized, varying electrographic signatures) or present with idiosyncratic EEG patterns during ictal events. The
Detach-MINIROCKET features, while robust, are learned from a diverse pool of seizures in the training set. However, if a particular subject's seizures show highly unique
or underrepresented electrographic characteristics compared to the dominant patterns in the training data, the model may struggle to generalize effectively to that specific
individual.

\subsection{Potential Variations in EEG Acquisition}

While both the TUSZ and Siena datasets adhere to the standard 10-20 electrode placement system, the practical application of scalp electrodes can introduce subtle yet
impactful deviations between subjects. Minor discrepancies in the precise anatomical landmarking for electrode positions can lead to considerable differences in the underlying
cortical sources captured by each channel~\cite{PASCUALMARQUI1999169}. Therefore, the spatio-temporal patterns learned by the Detach-MINIROCKET model might be expressed
differently across subjects due to placement variations. Additionally, inherent inter-subject differences in scalp thickness, skull conductivity, and electrode-scalp
impedance further modulate the recorded EEG signals~\cite{Lai2005Estimation}. These variations can persist even after careful preprocessing and would possibly influence
model performance on a subject basis.

\section{Limitations}
\subsection{Impact of Randomness on Model Performance and Reproducibility}
The performance variations observed across different experimental runs (See Table~\ref{tab:standardmodelperformance})in our study highlight a critical limitation inherent
to the ROCKET-based methodologies: the influence of multiple sources of randomness on model performance and result reproducibility. First, the random partitioning of
subjects into training and test sets introduces variability in the data distribution characteristics between different experimental runs. Given the substantial inter-subject
variability discussed in Section~\ref{sec:subjectvariations}, different subject splits can lead to training and test sets with markedly different difficulty levels and
distributional properties. Second, the core ROCKET methodology relies on the generation of numerous random convolutional kernels with stochastically determined parameters
including kernel length, weights, bias, dilation, and padding~\cite{Dempster2020rocket, Dempster2021mini}. The realization of these random kernels fundamentally determines
the feature space representation and, consequently, the model's capacity to capture relevant temporal patterns for seizure detection.

These randomness sources are shown as substantial variation in reported performance metrics across different experimental runs, even when using identical hyperparameters
and methodological configurations. Hyperparameter optimization and sensitivity analysis becomes less reliable when performance estimates are subject to high variance,
especially with limited experimantal runs. Additionally, the comparison between different methodological approaches may be confounded by random variation rather than reflecting
true algorithmic differences.

\subsection{Computational Resource Constraints and Methodological Limitations}

While the Detach-ROCKET and Detach-MINIROCKET models are inherently efficient and computationally lightweight compared to deep learning alternatives, our study was significantly
constrained by the large scale of the EEG datasets and the associated memory requirements rather than the computational complexity of the models themselves. The TUSZ dataset,
with its thousands of multi-channel EEG recordings, presents substantial challenges in terms of data loading, preprocessing, feature transformation storage, and overall
memory management, particularly when operating within a resource-constrained environment. The ensemble methodology, while computationally efficient in terms of individual
model training, amplifies the memory requirements due to the need to store and manage multiple model instances and their associated feature transformations.

Due to this limitation, we were constrained to use 50\% of the available subjects from the TUSZ dataset for training and testing. While the models demonstrated sufficient
performance on the test set(See Figure~\ref{fig:datasizemetrics}), this reduced dataset size could potentially limit their generalizability to unseen data.  Furthermore,
these memory limitations restricted our ability to conduct a more extensive exploration of the hyperparameter space, which may in turn affect our thorough understanding
of the effects these parameters have on model performance and behavior.

\section{Ethics and Sustainability}

The development and deployment of automated seizure detection systems raises important ethical considerations that must be carefully addressed to ensure responsible clinical
implementation. Our Detach-ROCKET-based approach demonstrates several positive characteristics that contribute to both ethical practice and environmental sustainability
in healthcare technology.

\subsection{Ethical Considerations}

The primary ethical imperative in developing automated seizure detection systems is to enhance patient care while minimizing potential harm. Our models are designed as
clinical decision support tools rather than replacements for neurological expertise. The interpretability features of the Detach Ensemble, particularly the channel relevance
estimation capabilities, provide clinicians with insights into the model's decision-making process, supporting informed clinical judgment rather than creating "black box"
automated decisions~\cite{solana2024classificationrawmegeegdata}.

The substantial inter-subject variability observed in our results (Figure~\ref{fig:subjectscoretusz}) highlights important considerations regarding algorithmic fairness
and bias. While some performance variation is inherent to the clinical heterogeneity of epilepsy, it is crucial that automated systems do not systematically disadvantage
particular patient populations. Our cross-dataset evaluation on the Siena database demonstrates the model's ability to generalize across different recording environments
and patient cohorts, suggesting reduced risk of systematic bias toward specific institutional or demographic groups.

Patient privacy and data security represent critical ethical concerns in EEG-based systems. The feature-based approach employed by Detach-ROCKET transforms raw EEG signals
into statistical representations that retain diagnostic utility while potentially reducing privacy risks compared to storing raw neurophysiological data. Additionally,
the computational efficiency of our approach enables local processing and analysis, reducing the need for sensitive medical data transmission and cloud-based computation.

\subsection{Sustainability Considerations}

The computational efficiency demonstrated by our Detach-ROCKET models contributes significantly to the environmental sustainability of automated seizure detection systems.
As shown in Table~\ref{tab:model_predict_time}, our single Detach-MINIROCKET model processes EEG recordings approximately 17 times faster than the catch22 benchmark,
while the ensemble approach remains nearly twice as fast. This computational efficiency translates directly into reduced energy consumption during clinical deployment,
particularly important given the continuous monitoring requirements typical of seizure detection applications.

The reduced computational requirements enable deployment on less powerful hardware platforms, extending the accessibility of automated seizure detection to resource-constrained
healthcare environments. This democratization of advanced neurological monitoring technology has significant implications for global health equity, particularly in low-
and middle-income countries where access to specialized neurological expertise may be limited.

Furthermore, the efficiency gains achieved through our methodology reduce the carbon footprint associated with large-scale EEG monitoring systems. Traditional deep learning
approaches often require substantial computational resources for both training and inference, contributing to significant energy consumption. The ROCKET-based feature transformation
approach requires minimal computational overhead during inference while maintaining competitive accuracy, representing a more sustainable approach to automated medical diagnosis.

The modular design of the Detach Ensemble also supports sustainable software development practices. The separation of feature generation from classification enables efficient
model updates and adaptations without requiring complete system retraining, reducing both computational costs and development time for clinical system maintenance and improvement.

\cleardoublepage
\chapter{Conclusions and Future work}
\label{ch:conclusionsAndFutureWork}

\section{Conclusions}
\label{sec:conclusions}

\end{document} 